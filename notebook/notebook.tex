\documentclass{article}
\voffset=-1.6in
\textheight=1.41\textheight
\hoffset=-.75in
\textwidth=1.1\textwidth
\usepackage{url}
\usepackage{tikz}
\usepackage{hyperref}
\usetikzlibrary{matrix,arrows} 
\include{macros}
\DeclareMathOperator{\sss}{ss}
\renewcommand{\ss}{\sss}
\renewcommand{\b}{\mathfrak{b}}
\renewcommand{\c}{\mathfrak{c}}
\newcommand{\cN}{\mathcal{N}}
\title{BSD Notebook}
\author{William Stein}
\date{}
\begin{document}
\maketitle
\tableofcontents

\section{August 28, 2009: Getting going}

I read the first 35 pages of Rockmore's book.  It is like 
fingernails on a chalkboard. 

RH book todo:
\begin{itemize}
 \item make a wiki page?
 \item get my hg repo up to speed
  \item read through text making list
\end{itemize}

Here is a todo list while reading the book:  
\begin{itemize}
\item Make a list of the books --both popular and not-- about
  RH. (mentioned on page 1).  I have Patterson (serious) and Edwards
  (serious) on my desk, and Sabbagh (popular) and Rockmore
  (cringe-inducing) on my desk too.  There could be other popular
  books that have chapters about RH that are good (or not).  E.g.,
  ``The Millennium Problems'' by Keith Devlin has chapter 1 about RH
  (the BSD chapter of that book sucks, but I maybe the RH chapter is
  good?).   
\item We say ``least mathematical background required'' but having
  tested our booklet on students, I would say that we do not succeed
  there.  We could make our booklet 2 times as long and require less
  math background.  I've slowly come to think this would be worth it.
  And we could make it much longer still by adding way more
  illustrations (generated by Sage) and lots of prose explaining what
  is in the illustrations (little guided tours), and this would also
  be worth it.

\item I definitely want to say more in the book about how RH informs
  complexity analysis in computational number theory.  This is perhaps
  the main way RH appears in modern computational number theory.
  Maybe there are some very simple down-to-earth examples of this
  principle at work.

\item I would almost like to restructure things so the illustrations
  are much more extensive and integral and included in the main text.
  Then the additional Sage interacts are merely an ``additional
  resource'' for those wishing to investigate further.  They're an
  added bonus.  But they can also be safely ignored. 

\item Typo: ``websitte''.

\item I think we should remove the business about how long it should
  take to read the book.  Let the reader start reading and decide for
  themselves.  Otherwise, they might feel insecure and wonder all the
  time if they are taking ``too long''. 

\item ``less than 100, 10,000, 1,000,000, `` that looks at first
  glance like a single huge number.  Maybe make it three statements.
less than 100?  less than 10,000?  less than 1,000,000?

\item Picture of Bott?

\item Picture of Zagier?

\item This sentence: ``If we are to believe Aristotle, the early
  Pythagoreans thought that the principles governing Number are “the
  principles of all things,” the elements of number being more basic
  than the Empedoclean physical elements earth, air, fire, water.''
  I've been looking at other popular math books, and they never just
  assume the reader knows who Aristotle is, Pythagoreans were, or what
  Empedoclean means.  In fact, I have no idea what Empedoclean means,
  and I can easily forgot {\em when} the Pythagoreans were around. 
That said, I would rather say nothing to say something wrong. 
  
\item Descarte picture?  Are there any?  [[Yes -- see Wikipedia]]

\item Speaking of ``wrong'' (see above), somebody emailed me this:
\begin{verbatim}
In a text "Elementary Number Theory" in section 7.1.2, you have
an implementation of the sieve of Eratosthense.   Melissa O'Neill
wrote a paper, "The Genuine Sieve of Eratosthenese".  I do not
believe that your program meets here criteria for being the
genuine sieve of Eratosthenses.   I used IDLE on an IBM/PC to
run your program and crashed, if I entered the value of 200000.
I can create a list of primes at least up to 200000 if I use a
Python program that meets her criteria.
\end{verbatim}
We should read   Melissa O'Neill's paper to see what the deal is.

\item ``Contemporary physicists dream of a “final theory.”'' Do they
  really?  In what sense?

\item ``Don Quixote encountered this...'': Who is he exactly?  A
  fictional character, a person?  When?  I've heard of him, but
  honestly I've never read anything nontrivial about him, and I doubt
  most of our readers will have.  They might see him mainly as a
  mysterious person whose last name is hard to pronounce.

\item Why exactly do Cicada's come out every 17 years?  I saw Bruce
  Jordan in Princeton recently and we started talking about this
  (since they have Cicada's there), and I quickly realized I didn't
  really have a clue.

\item ``Philolaus (a predecessor of Plato)'' that isn't a good enough
  introduction to Philolaus, given that it is the first mention of
  Plato.  Again, many readers might not know Plato so well. Heck, I
  don't.  I view all the above remarks as opportunities to expand our
  book's readership and mission a bit, rather than criticisms of it. 

\item ``But, until Euclid, prime numbers seem not to have been singled
  out as the extraordinary math- ematical concept, central to any deep
  understanding of numerical phenomena, that they are now understood
  to be.''  Here we are foreshadowing Euclid's proof that there are
  infinitely many primes, etc.  But this is also the first time Euclid
  is even mentioned.  To a casual reader it just feels that it's a
  point in an outline that hasn't been filled in.

\item Instead of starting with the 300 factoring example, perhaps we
  should first start with a smaller one where we can list {\em every}
  single factorization tree/order.  This makes things feel less {\em
    abstract}, since the reader doesn't have to imagine all the
  missing factorizations.

\item ``more than 100 digits, to your computing machine and ask it to
  multiply them together: you will get their product N = P × Q with
  its 200 or so digits in a few microseconds.''  I just checked and it
  is a few hundred {\em nanoseconds} to do that.  So lets change to
  ``few hundred nanoseconds'' or perhaps better ``a just under a
  microsecond.''


\item Our proof of the infinitude of primes on page 8 is the first
  time in the book we use symbolic notation, give a proof, reason
  abstractly, etc.  I wonder if we could do a little more to prepare
  the reader.  I just read Rockmore's horrendous proof of the same
  thing in his book on RH -- it's pages of tedium to say in words what
  takes 1 second with symbols.  But I'm attracted to the challenge of
  doing something a little bit in between, e.g., having an example.

\item Move our discussion of EFF cash prize up, since the prize was
  just won!  I wonder if there are any press releases about the prize
  being awareded, which we could cite or point to?

\item Here is Sage actually computing the decimal digits of the
  biggest known Mersenne prime:
\begin{verbatim}
sage: time a =2^43112609-1
CPU times: user 0.01 s, sys: 0.01 s, total: 0.01 s
Wall time: 0.02 s
sage: time s = str(a)
CPU times: user 12.23 s, sys: 0.99 s, total: 13.22 s
Wall time: 13.63 s
sage: s[-10:]
'6697152511'
sage: time sum(a.digits())
CPU times: user 15.25 s, sys: 1.07 s, total: 16.33 s
Wall time: 16.84 s
58416637
\end{verbatim}

\item ``But there is no obvious way'' -- maybe ``no known way''?

\item ``In Figure 3.3 we use the primes 2, 3, 5, and 7 to sieve out
  the primes up to 100, where instead of crossing out multiples we
  grey them out, and instead of circling primes we color their box
  red.''  I could make a sequence of figures where we do cross them
  out too?  The grey background is hard to see and probably hard to
  print, so I can do better there too.

\item For all these questions: ``Are there infinitely many pairs of
  primes whose difference is 4? Answer: equally unknown. Is every even
  number greater than 2 a sum of two primes? Answer: unknown. Are
  there infinitely many primes which are 1 more than a perfect square?
  Answer: unknown.''  we could give precise references into Richard
  Guy's book ``Unsolved Problems in Number theory'', which in turn has
  a very good collection of references and more detailed description
  of each problem.  This would be a good endnote. 

\item I may as well draw a plot of $\text{Gap}_k(X)$ for various k together
on one plot. 

\item We have several natural {\em sections} already in the first 8
  pages, but don't break them up as such.  We should.  It would make
  things easier to navigate.  We have a section ``what are primes''.  Then ``prime gaps''.  Then ``multiplicative parity''.

\item ``Here is some data:''... and a weird big page break?

\item On page 12, the references to Borwein etc. should of course be
  moved to an endnote.

\item On page 15 (Fig 5.4), it would be nice to have a less zoomed out
  big figure for starters, to look at while reading along.  Basically
  like Fig 5.5, which looks very nice.  Those figures could be bigger
  too.  I really like this part of the text though, where we are
  spending a lot of care explaining the mathematics.

\item I need to figure out how to be very precise in placing all the
  figures where we want, not where latex wants.  Right now there
  placement significantly detracts from readability.

\item ``The particular issue before us is, in our opinion, twofold,
  both applied, and pure: can we curve-fit the “staircase of primes”
  by a well approximating smooth curve?''  I think it would be worth
  emphasizing that our smooth curve must be given for a ``formula''.
  I mean, a typical reader might just think ``of course any kid could
  take a pencil and draw a smooth curve through the staircase of
  primes''.  But to get a curve given by an sort of analytic formula
  at all and which happens to have {\em anything } at all to do with
  the function $\pi(X)$ -- well that seems really hard.  A typical
  reader might have no idea where to start to do that.  Maybe we can
  express this somehow?  

\item `` the chances that a number N is a prime is inversely
  proportional to the number of digits of N''. Does there exist any
  heuristic plausibility argument for this assumption that would make
  sense to give at this point?  Things are made a bit confusing since
  the constant isn't 1, e.g., the probability that a number around a
  billion is prime is not about ``1 in 9''.

\item Pure and applied math.  I think we should double the length of
  this section by adding some examples.  In particular, we could use
  examples all of which will appear later!  Examples of possible
  illustrations include:
\begin{itemize}
  \item something foreshadowing Fourier theory (applied)
  \item random walks (finance?)  (applied)
  \item data compression (applied)
  \item Mersenne primes, e.g., the Lucas-Lehmer test for primality (pure mathematics)
  \item Goldbach's conjecture (pure)
  \item The Hardy-Littlewood conjecture about asymptotics of $\text{Gap}_k(X)$. (pure)
  \item Complex numbers (pure and applied); and with an endnote that
    points at your book Imagining Numbers?
  \end{itemize}
  
\item 

\end{itemize}





\end{document}
