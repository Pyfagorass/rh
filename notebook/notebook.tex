\documentclass{article}
\voffset=-1.6in
\textheight=1.41\textheight
\hoffset=-.75in
\textwidth=1.1\textwidth
\usepackage{url}
\usepackage{tikz}
\usepackage{hyperref}
\usetikzlibrary{matrix,arrows} 
\include{macros}
\DeclareMathOperator{\sss}{ss}
\DeclareMathOperator{\Li}{Li}
\renewcommand{\ss}{\sss}
\renewcommand{\b}{\mathfrak{b}}
\renewcommand{\c}{\mathfrak{c}}
\newcommand{\cN}{\mathcal{N}}
\title{BSD Notebook}
\author{William Stein}
\date{}
\begin{document}
\maketitle
\tableofcontents

\section{August 28, 2009: Getting going}

I read the first 35 pages of Rockmore's book.  It is like 
fingernails on a chalkboard. 

RH book todo:
\begin{itemize}
 \item make a wiki page?
 \item get my hg repo up to speed
  \item read through text making list
\end{itemize}

Here is a todo list while reading the book:  
\begin{itemize}
\item Make a list of the books --both popular and not-- about
  RH. (mentioned on page 1).  I have Patterson (serious) and Edwards
  (serious) on my desk, and Sabbagh (popular) and Rockmore
  (cringe-inducing) on my desk too.  There could be other popular
  books that have chapters about RH that are good (or not).  E.g.,
  ``The Millennium Problems'' by Keith Devlin has chapter 1 about RH
  (the BSD chapter of that book sucks, but I maybe the RH chapter is
  good?).   
\item We say ``least mathematical background required'' but having
  tested our booklet on students, I would say that we do not succeed
  there.  We could make our booklet 2 times as long and require less
  math background.  I've slowly come to think this would be worth it.
  And we could make it much longer still by adding way more
  illustrations (generated by Sage) and lots of prose explaining what
  is in the illustrations (little guided tours), and this would also
  be worth it.

\item I definitely want to say more in the book about how RH informs
  complexity analysis in computational number theory.  This is perhaps
  the main way RH appears in modern computational number theory.
  Maybe there are some very simple down-to-earth examples of this
  principle at work.

\item I would almost like to restructure things so the illustrations
  are much more extensive and integral and included in the main text.
  Then the additional Sage interacts are merely an ``additional
  resource'' for those wishing to investigate further.  They're an
  added bonus.  But they can also be safely ignored. 

\item Typo: ``websitte''.

\item I think we should remove the business about how long it should
  take to read the book.  Let the reader start reading and decide for
  themselves.  Otherwise, they might feel insecure and wonder all the
  time if they are taking ``too long''. 

\item ``less than 100, 10,000, 1,000,000, `` that looks at first
  glance like a single huge number.  Maybe make it three statements.
less than 100?  less than 10,000?  less than 1,000,000?

\item Picture of Bott?

\item Picture of Zagier?

\item This sentence: ``If we are to believe Aristotle, the early
  Pythagoreans thought that the principles governing Number are “the
  principles of all things,” the elements of number being more basic
  than the Empedoclean physical elements earth, air, fire, water.''
  I've been looking at other popular math books, and they never just
  assume the reader knows who Aristotle is, Pythagoreans were, or what
  Empedoclean means.  In fact, I have no idea what Empedoclean means,
  and I can easily forgot {\em when} the Pythagoreans were around. 
That said, I would rather say nothing to say something wrong. 
  
\item Descarte picture?  Are there any?  [[Yes -- see Wikipedia]]

\item Speaking of ``wrong'' (see above), somebody emailed me this:
\begin{verbatim}
In a text "Elementary Number Theory" in section 7.1.2, you have
an implementation of the sieve of Eratosthense.   Melissa O'Neill
wrote a paper, "The Genuine Sieve of Eratosthenese".  I do not
believe that your program meets here criteria for being the
genuine sieve of Eratosthenses.   I used IDLE on an IBM/PC to
run your program and crashed, if I entered the value of 200000.
I can create a list of primes at least up to 200000 if I use a
Python program that meets her criteria.
\end{verbatim}
We should read   Melissa O'Neill's paper to see what the deal is.

\item ``Contemporary physicists dream of a “final theory.”'' Do they
  really?  In what sense?

\item ``Don Quixote encountered this...'': Who is he exactly?  A
  fictional character, a person?  When?  I've heard of him, but
  honestly I've never read anything nontrivial about him, and I doubt
  most of our readers will have.  They might see him mainly as a
  mysterious person whose last name is hard to pronounce.

\item Why exactly do Cicada's come out every 17 years?  I saw Bruce
  Jordan in Princeton recently and we started talking about this
  (since they have Cicada's there), and I quickly realized I didn't
  really have a clue.

\item ``Philolaus (a predecessor of Plato)'' that isn't a good enough
  introduction to Philolaus, given that it is the first mention of
  Plato.  Again, many readers might not know Plato so well. Heck, I
  don't.  I view all the above remarks as opportunities to expand our
  book's readership and mission a bit, rather than criticisms of it. 

\item ``But, until Euclid, prime numbers seem not to have been singled
  out as the extraordinary math- ematical concept, central to any deep
  understanding of numerical phenomena, that they are now understood
  to be.''  Here we are foreshadowing Euclid's proof that there are
  infinitely many primes, etc.  But this is also the first time Euclid
  is even mentioned.  To a casual reader it just feels that it's a
  point in an outline that hasn't been filled in.

\item Instead of starting with the 300 factoring example, perhaps we
  should first start with a smaller one where we can list {\em every}
  single factorization tree/order.  This makes things feel less {\em
    abstract}, since the reader doesn't have to imagine all the
  missing factorizations.

\item ``more than 100 digits, to your computing machine and ask it to
  multiply them together: you will get their product N = P × Q with
  its 200 or so digits in a few microseconds.''  I just checked and it
  is a few hundred {\em nanoseconds} to do that.  So lets change to
  ``few hundred nanoseconds'' or perhaps better ``a just under a
  microsecond.''


\item Our proof of the infinitude of primes on page 8 is the first
  time in the book we use symbolic notation, give a proof, reason
  abstractly, etc.  I wonder if we could do a little more to prepare
  the reader.  I just read Rockmore's horrendous proof of the same
  thing in his book on RH -- it's pages of tedium to say in words what
  takes 1 second with symbols.  But I'm attracted to the challenge of
  doing something a little bit in between, e.g., having an example.

\item Move our discussion of EFF cash prize up, since the prize was
  just won!  I wonder if there are any press releases about the prize
  being awareded, which we could cite or point to?

\item Here is Sage actually computing the decimal digits of the
  biggest known Mersenne prime:
\begin{verbatim}
sage: time a =2^43112609-1
CPU times: user 0.01 s, sys: 0.01 s, total: 0.01 s
Wall time: 0.02 s
sage: time s = str(a)
CPU times: user 12.23 s, sys: 0.99 s, total: 13.22 s
Wall time: 13.63 s
sage: s[-10:]
'6697152511'
sage: time sum(a.digits())
CPU times: user 15.25 s, sys: 1.07 s, total: 16.33 s
Wall time: 16.84 s
58416637
\end{verbatim}

\item ``But there is no obvious way'' -- maybe ``no known way''?

\item ``In Figure 3.3 we use the primes 2, 3, 5, and 7 to sieve out
  the primes up to 100, where instead of crossing out multiples we
  grey them out, and instead of circling primes we color their box
  red.''  I could make a sequence of figures where we do cross them
  out too?  The grey background is hard to see and probably hard to
  print, so I can do better there too.

\item For all these questions: ``Are there infinitely many pairs of
  primes whose difference is 4? Answer: equally unknown. Is every even
  number greater than 2 a sum of two primes? Answer: unknown. Are
  there infinitely many primes which are 1 more than a perfect square?
  Answer: unknown.''  we could give precise references into Richard
  Guy's book ``Unsolved Problems in Number theory'', which in turn has
  a very good collection of references and more detailed description
  of each problem.  This would be a good endnote. 

\item I may as well draw a plot of $\text{Gap}_k(X)$ for various k together
on one plot. 

\item We have several natural {\em sections} already in the first 8
  pages, but don't break them up as such.  We should.  It would make
  things easier to navigate.  We have a section ``what are primes''.  Then ``prime gaps''.  Then ``multiplicative parity''.

\item ``Here is some data:''... and a weird big page break?

\item On page 12, the references to Borwein etc. should of course be
  moved to an endnote.

\item On page 15 (Fig 5.4), it would be nice to have a less zoomed out
  big figure for starters, to look at while reading along.  Basically
  like Fig 5.5, which looks very nice.  Those figures could be bigger
  too.  I really like this part of the text though, where we are
  spending a lot of care explaining the mathematics.

\item I need to figure out how to be very precise in placing all the
  figures where we want, not where latex wants.  Right now there
  placement significantly detracts from readability.

\item ``The particular issue before us is, in our opinion, twofold,
  both applied, and pure: can we curve-fit the “staircase of primes”
  by a well approximating smooth curve?''  I think it would be worth
  emphasizing that our smooth curve must be given for a ``formula''.
  I mean, a typical reader might just think ``of course any kid could
  take a pencil and draw a smooth curve through the staircase of
  primes''.  But to get a curve given by an sort of analytic formula
  at all and which happens to have {\em anything } at all to do with
  the function $\pi(X)$ -- well that seems really hard.  A typical
  reader might have no idea where to start to do that.  Maybe we can
  express this somehow?  

\item `` the chances that a number N is a prime is inversely
  proportional to the number of digits of N''. Does there exist any
  heuristic plausibility argument for this assumption that would make
  sense to give at this point?  Things are made a bit confusing since
  the constant isn't 1, e.g., the probability that a number around a
  billion is prime is not about ``1 in 9''.

\item Pure and applied math.  I think we should double the length of
  this section by adding some examples.  In particular, we could use
  examples all of which will appear later!  Examples of possible
  illustrations include:
\begin{itemize}
  \item something foreshadowing Fourier theory (applied)
  \item random walks (finance?)  (applied)
  \item data compression (applied)
  \item Mersenne primes, e.g., the Lucas-Lehmer test for primality (pure mathematics)
  \item Goldbach's conjecture (pure)
  \item The Hardy-Littlewood conjecture about asymptotics of $\text{Gap}_k(X)$. (pure)
  \item Complex numbers (pure and applied); and with an endnote that
    points at your book Imagining Numbers?
  \end{itemize}
  
\item Picture of Gauss  (and I really like our Gauss dates)

\item ``Roughly speaking, this means that the number of primes up to X
  is X times the reciprocal of 2.3 times the number of digits of X .''
  I think this is confusing to read.  The reciprocal of 2.3 is kind of
  funny, since 2.3 is already mysterious.  It's really $1/log_e(10)$,
  which is $0.43429448190325176...$, or basically $.4$.  Maybe better
  would be ``Very roughly speaking, this means that the number of
  primes up to $X$ is about $X$ divided by twice the number of digits
  of $X$.''  We can make a
  table to illustrate this further, but also to emphasize that it's
  not that close. Something like this:
\begin{verbatim}
sage: for i in [2..10]: print i, prime_pi(10^i), floor((10^i-1)/(i*2))
....: 
2 25 24
3 168 166
4 1229 1249
5 9592 9999
6 78498 83333
7 664579 714285
8 5761455 6249999
9 50847534 55555555
10 455052511 499999999
\end{verbatim}

  It's kind of convenient that for $99$, $999$, and $9999$, the
  approximations got by taking ``$X$ divided by twice the number of
  digits of $X$ are very close to $\pi(X)$.  Anyway, rounding to $0.5$
  instead of $0.43$ makes it really simple to describe.

\item We might do something to warn our reader that if they see
  ``$\log(X)$'' they shouldn't run in fear and think ``holy crud, I
  have no idea what log is and I never understand that in high
  school,'' since we are about to explain it.  I'm imagining say my
  brother as reading this -- he literally probably hasn't seen log
  once in a decade though he is good with numbers (running five
  businesses in San Diego).  He told me that when he sees a page with
  a mathematical formula involving symbols he doesn't know, he'll just
  block it out.  So if we sneak ``log'' in to a sentence or two before
  we use it in a formula, it'll get by that filter (which is probably
  pretty common with non-math people).   

I'm imagining a solution like this:
\begin{itemize}
\item We figure out where logs first came from and give one sentence about
this (I think they arose in doing arithmetic efficiently?)  

\item We demystify log {\em before} using it in any formula by explaining
that it is ``about twice the number of digits''. 

\end{itemize}

One other issue is that in much of math education, unfortunately $\ln$
means natural log and $\log$ means $\log$-base-10.  It's really
annoying... We can mention this somehow. 

It might be worthwhile to remark that $e^x$ is the unique nonzero
function that equals its own derivative -- perhaps this is a way to
sneak in a mention of derivatives before later in the book where we
use them a lot more.  Anyway there are two issues: (1) what are logs,
and (2) what is this ``natural log''?

\item ``the 2004 US elections'' -- this will not be in people's minds
  for a {\em book} so much.  It may be better to remove or expand with
  a statement about just how close they were with a reference.  E.g.,
  ``the 2004 US elections, in which ... beat ... by a mere
  ... votes!''  Wasn't the 2000 election even closer, or am I
  mis-remembering?

\item ``So when Gauss thought his curve missed...''  let's compute the
  square root explicitly here, i.e., just spell this out some more
  (instead of leaving an exercise for the reader).

\item ``devil fable''  I found this graphic via an image search on google: 

\url{http://blog.al.com/stantis/2007/11/Stantis-Devil%20in%20the%20Langford%20details.jpg}

If the cover were modified, or the whole thing redrawn, it could be fun. The cover could say $primes up to X$.  Or it could be replaced by the checklist...

\item We write ``$\pi(X)$ for various large numbers $N$'' in our devil fable. Oops.

\item We should draw an illustration of the checklist in the story.
  It would be easy.

\item We make the claim ``The average error (over-counted or
  undercounted) would be proportional to $\sqrt{N}$.''  We do not
  justify this claim at all.  We might say that it follows from a
  result about random walks.  (Does it really follow from the central
  limit theorem somehow?)  Also, given that we assumed that the error
  rate is 0.001\% can't we say what the constant in the proportion is?
  Also, I think we could give an estimate of how far they would be off
  for $N=3,000,000$.  We could deduce Gauss's error rate, right?

\item In figure 10.1 with plots of Li, pi, and X/Log(X), I should
put labels in the actual plot.  It is lazy putting them only
in the caption.

\item I should update the $X=4\cdot 10^{22}$ to whatever
the current record is, I think $10^{24}$, maybe.  And also update
the reference, which may be wrong.  Also, here is where we can possibly
discuss how to compute $\pi(X)$, or if not we can at least point
to an (extended) endnote.  When this is done, be sure to search and
update all other references to $4\cdot 10^{22}$.

\item We write -- `` an easier fact, which follows directly from
  elementary calculus'' for the fact that $\Li(x)$ is asymptotic to
  $X/\log(X)$. We should prove this rigorously in an endnote.

\item ``It was proved in 1896 indepdently by Hadamard and de la Valle Poussin. ``

(1) typo in  ``indepdently'';  we should say something about who these guys are, 
and give links to Wikipedia (say).

\item We write ``is much deeper than the Prime Number Theorem''.  I
  think the phrase ``is much deeper'' is mathematical jargon, because
  popular math books would often have a little interlude to say
  something about what deep means to mathematicians.  It's basically
  ``difficult and any proof will use and influence a wide range of
  mathematics''.  So we too can add a little more to emphasize what we
  mean by the word ``deep''.  Or we can just say that the rest of this
  paragraph explains what we mean (indeed, it does).  Maybe everything
  is perfect as is.

\item ``It is the kind of conjecture that Frans Oort...'' let's have a
  sentence about who Oort is.  E.g., Dutch mathematician, born 19xx,
  student of xxx...  I might have a picture of him too.  

\item We write ``A proof of RH would, therefore, fall into the applied
  category, given our discussion above.'' But we changed our
  discussion above, so this is no longer quite true.

\item I wonder if I could draw a 3d picture of an actual staircase
  whose side profile is the plot of $\pi(X)$, but rendered at an angle
  to look like a real staircase.  This might be a nice illustration.

\item In the section ``Tinkering with the carpentry of the staircase
  of primes.''  I should draw several plots illustrating every single
  one of the steps we discuss about tinkering with the staircase.

\item ``These vertical dimensions might lead to a steeper ascent but
  no great loss of information'' Maybe change to ``Since $\log(p)>1$,
  these vertical dimensions lead to a steeper ascent but no great loss
  of information.''

\item ``Do not worry if you do not understand why our first and second
  formulations of Riemann's Hypothesis are equivalent.''  We should
  either rigorously prove this in an endnote (my preference at the
  moment) or gave a reference that totally does it.  I could imagine a
  better student who has a more advanced background, who would benefit
  by seeing a proof at this point.  And it might help us keep things
  straight... e.g., we had this equivalence wrong I think in some
  version of our notes long ago.

\item ``variety of equivalent ways we have to express Riemann’s
  propose answers to the question'' -- I think ``propose'' should be
  ``proposed''.

\item I'm worried that our second statement of RH is possibly
  confusing because it says ``This new staircase is essentially square
  root close''.  However, given a line and curve the notion of close
  is vague.  What we really mean is that the function $\psi$ given by
  the new staircase is an essentially square root approximation to the
  function $f(x) = x$.

\item Having just read ``Tinkering with the carpentry of the staircase
  of primes.'' I think it starts out mysteriously.  I think we should
  start with a paragraph that the point of the work (really, it feels
  like some serious manual labor with all the carpentry)! is to give
  an equivalent formulation of RH that simply asserts that a certain
  function that we will construct from counting prime powers is an
  essentially square root approximation to $f(x)=x$.

\item I wonder if we should say something right before stating RH 2
  about what it means for two mathematical statements to be
  equivalent?  Equivalence of statements is a sort of critically
  important basic tool in all of mathematical research, and is
  something students encounter early on when simplifying expressions
  and doing algebra.  It permeates math.  We touch on this also when
  mention the multiplicative parity situation, where instead of giving
  an equivalent statement, we give a statement that might {\em a
    priori} be equivalent, but which turns out to only imply RH.
  Anyway, I think there is an opportunity here.

\item What are the frequence and amplitudes of pure C and E notes?
We could say a concrete illustrations of what we're talking about 
in the section ``What do computer music files, data-compression, and prime 
numbers have to do with each other?''

\item ``But this sampling would take an enormous amount of storage
  space!''  Well it would if you sampled at too many points.  We might
  say that to sound good it takes about xxx samples {\em per second}.
  (Give the rate for audio CD's).  Heh, we do say that, so rewording
  this slightly might help.  It might be nice to say how much space
  44khz takes up, since CD's are actually uncompressed.  We could say
  that we're explaining why an audio CD has only about 12 sonds on it,
  but exactly the same audio CD can easily hold 100 MP3's.  (We say
  this later...)

\item ``Surprisingly, this seems to be roughly the way our ear
  processes such a sound when we hear it.'' (in reference to storing
  the spectrum, etc.)  Is this a {\em biological} statement, and if so
  is it the result of some research in biology that we could cite?
  Otherwise, where does this assertion come from?  Having us two
  authors and explaining (possibly with footnotes) where all our
  assertions come from I think will make our book vastly more solid
  than most popular math books, which are often just full of seemingly
  random unjustified statements.

\item ``At this point we recommend to our readers that they
  download...'' However, we don't recommend that they read it right
  now!  They should finish our book first. :-) I want our book to be a
  pager turner that they can't put down.  That they blow off
  everything so they can finish reading it.  Actually, because of
  that, we should maybe put in more foreshadowing at the beginning of
  this section and throughout.  I want something like the paragraph at
  the end of section 13 full of questions (top of page 33), but at the
  beginning of section 13.

\item (random comment) I love the idea of putting all the distracting
  links in endnotes -- I'm imagining a reader that plows through our
  whole book, not putting it down, not looking at footnotes, then says
  ``I want to read that again'', and only on a second reading of
  certain parts really dives into the footnotes.  Your ``Imagining
  Numbers'' book was exactly like that and I think it really works.
  Many popular math books are not, and it is very frustrating reading
  them as a result (and they are often strongly criticized for just
  this in Bulletins/Notices reviews, I think).

\item ``So our CE chord'' -- do musicians really write ``CE'' to mean
  ``some combination of C and E''?  I don't know.  If so, we might say
  ``musicians write CE to mean ...''  If not, what do they write?
  Should their be some notes (you know like what musicians actually
  definitely do write) somewhere on our page?

\item I think we should give lots more examples in the text like Fig
  13.10 and Fig 13.11 and explain maybe something about why some of
  them are valid (?).   

\item ``psycho-acoustic understanding.'' replace by a sentence saying
  what that is, e.g., that humans only here certain frequencies
  (etc.).  Also, in that paragraph we could emphasize that a factor of
  10 in compression is revolutionary -- it means you get 100 songs on
  a CD instead of 12, and 200 albums on your ipod instead of 20.

\item I wish we could end section 13 with {\em something} more, even
  if it isn't at all technical. What about an illustration like Fig
  18.4 (on page 45) and some sort of clever language that -- in a
  nontechnical way -- explains it.  It's a vivid picture.  That image
  shows something that looks like sound waves, and it has primes in
  it.  That image might be on the cover of our book.  How close can we
  get to it in Part 1???

\item The Calculus Fig 15.1 -- yep, replacing it by log makes a lot of sense.

\item We could also give a plot of a wiggly polynomial, maybe $2x^3 -
  7x^2 + 5x - 2$ and its derivative $6x^2 - 14x + 5$, and note the
  remarkable pattern that the derivative is got from the original
  function in this case by reducing the exponents by $1$, etc.  We
  could remark that general observations just like this are a major
  theme in calculus.  

\item Give more examples of derivatives of functions, many of which
  we'll end up using later.  Example derivative of constant function,
  derivative of a line, derivatives of trig functions, etc. 

\item In Fig 15.2 (the graph that jumps) the axes labels are tiny.

\item Who are this guy?:  ``Newton and/or Leibniz''.

\item ``Notice, what is happening:''  Delete the comma?

\item Add an endnote and reference(s) for the paragraph on page 35
  about distributions.  What is a good reference (or references) for a
  student to turn to?

\item (**) I wonder if we should say what a ``function''.  We're spending a
  lot of energy saying that $\delta$ {\em isn't} a function, but we
  didn't say what a function is.  I didn't know an official definition
  of function until my third year of undergraduate school, so the
  target audience I have in mind doesn't know an ``official
  definition'' either.  In Calculus one typically sees sloppy things
  like ``the function $1/x$ which is infinite at $0$'', so for us to
  go on about delta not being a function because it is infinite is
  disigeneous.  Also, often us mathematicians do consider functions
  $\R \to \R\cup\{\infty\}$, say.  It occurs to me that the problem
  with $\delta$ is not that it is not a function, but that it is not a
  function that behaves well with respect to Calculus.  The $\delta$
  distribution is much better since e.g. $\int f \delta$ behaves so
  sensibly.

\item The caption for Figure 15.4 is totally wrong. It says ``A
  picture of the derivative of a smooth graph approximating the graph
  that is 1 up to some point and then 0 after that point. In each
  case, the blue graph is 1 until 1 ε and 2 after 1 + .''
Wrongness:  It's 4 pictures, not 1;  It doesn't immediately jump from 0 to 1;
what is a ``smooth graph''?     Etc. It just seems sloppy/wrong.

\item ``Continuous approximation to the staircase $\Psi(x)$ (in red)
  along with a plot (in blue) of the derivative of this [[insert 'continuous']]
  approximation''

\item (**) ``As we have hinted above, we lose no information if we
  further modify our staircase by distorting the $x$-axis, replacing
  $x$ by $e^t$''.  We could go way slower here, and have a few
  paragraphs (?)  about deforming the $x$ axis by a function.  We
  could give several examples, pictures, etc., just like we did for
  adding together two pure sounds, and I think it would help greatly
  to clarify what is going on.  Let's give a good specific picture and
  catalogue of examples to illustrate composition functions and
  thinking about what happens to their graphs.  Also, in the
  particular case of composing with $e^t$ isn't this just the
  incredibly-familiar-in-science process of plotting data on a log
  scale (or maybe exponential scale)?  Every science student has
  probably seen that, so it's definitely worth making that connection.


\item We through in a factor of $e^{t/2}$ in addition to precomposing
with $e^t$.  It seems like we do the division by $e^{t/2}$ without even
commenting on that.  Let's fix that.  Why is it there?

\item ``We will refer to distributions with discrete support as spike
  distributions'' This language is vivid and I like it for our
  booklet.  Let's add an endnote though that gives the standard
  terminology (e.g., ``discrete distrubtion'').

\item ``But there are many other ways to package this vital
  information, so we must explain our motivation'' If we had some even
  better versions of pictures like maybe Fig 18.4 or maybe 17.4 way
  earlier, we might say that explaining them is a big part of our
  motivation.  This is a little bit circular... but it is actually
  honestly what {\em our} motivation was when we wrote this stuff up
  in the first place.

\item I'm still curiuos how people knew that Riemann computed
zeros...

\item When we did this project initially there was a lot of excitement
  as we figured out exactly who to make Figure 18.4.  I think our
  current text fails to convey that excitement, but I really {\em
    want} to convey it.  I think it's worth making our book longer and
  spending more time explaining things.    

\item It is also sad that we moved $R(x)$ out of the main text.  I
  guess I really want part 3 back, or to make part 2 longer.  I
  thinked we broke it out last time due to fatigue?  I'm not fatigued
  anymore.

\item ``That a simple geometric property of these zeroes...'' this
  paragraph must come back into our book.  It's a very nice ending, or
  maybe a good motivation in the middle.

\item ``Our aim is not even to mention complex numbers in the text,''
  I'm suddenly wondering if this is at all a good constraint to put on
  ourselves.  I learned about -- and understood -- complex numbers in
  school in 8th grade, and that was in a {\em very} small country town
  in Texas.  I suspect a lot of people know the basics of complex
  numbers. People don't know complex {\em analysis} though.  Maybe can
  occassionally allow ourselves complex numbers, but definitely not
  complex analysis?


\end{itemize}



\section{August 20, 2009}

We have $\Phi(t) = \Psi'(e^t)/e^{t/2}$, and computed that the even and odd
Fourier transforms are:

$${\hat \Phi}_{\rm even}(s): = \sum_{p^n} {\frac{\log(p)}{p^{n/2}}}\cos(ns \log(p))$$ 
and the odd Fourier transform is
$${\hat \Phi}_{\rm odd}(s): = \sum_{p^n} {\frac{\log(p)}{p^{n/2}}}\sin(ns \log(p)).$$
   
   
Plotting these to low precision (with few $p$) and one clearly "sees" the zeros of zeta influencing
the plot.   Why?  This seems completely mysterious.   To much higher precision this disappears.
Barry suggests that instead we note that really say
${\hat \Phi}_{\rm even}(s)$ is really a distribution so it only makes sense to
integrate it against compact functions.  In fact, we should have
$$
   {\hat \Phi}_{\rm even}(s) = \sum_{\theta_i} \delta_{\theta_i}
$$
i.e., ${\hat \Phi}_{\rm even}(s)$ is the sum of Dirac deltas at the imaginary parts
of the zeros of the Riemann zeta function.

Recall that $\delta_a$ is characterized by 
$$
\int f(x) \delta_a dx = f(a).
$$

So, if $f(s)$ is a function with compact support, then 
$$
  \int f(s) {\hat \Phi}_{\rm even}(s) = \sum_{\theta_i} f(\theta_i).
$$
Thus if we consider a function $f(s)$ with compact support in a set $S$, then
$$
  \int f(s) {\hat \Phi}_{\rm even}(s) = \sum_{\theta_i\in S} f(\theta_i).
$$
E.g., suppose $S=[14,15]$.  Then 
$$
  \int f(s) {\hat \Phi}_{\rm even}(s) = f(\theta_0).
$$
(all the integrals above are from $-\infty$ to $\infty$).
 Let's say the interval $[a,b]$ contains only $\theta_0$, say.
Then 
$$
\int_{\infty}^{\infty} f(s) {\hat \Phi}_{\rm even}(s)
 = 
\int_{a}^{b} s {\hat \Phi}_{\rm even}(s)
= \int_{a}^{b} \sum_{p^n} {\frac{\log(p)}{p^{n/2}}} s\cos(ns \log(p))  ds.
$$
It seems natural to reverse the order of integration and summation:
$$
\int_{a}^{b} \sum_{p^n} {\frac{\log(p)}{p^{n/2}}} f(s)\cos(ns \log(p))  ds
 = 
 \sum_{p^n} {\frac{\log(p)}{p^{n/2}}} \int_{a}^{b} f(s)\cos(ns\log(p)) ds.
$$
So if we arrange that $\int_{a}^{b} f(s)\cos(s\log(p^n)) \to 0$ as
$p^n\to\infty$, we would be set.   But is there any way to do this?


I've tried for a while, and can't find an $f(s)$ so that this converges.

But I'm just really {\em amazed} by the mystery that 

$${\hat \Phi}_{\rm even}(s): = \sum_{p^n} {\frac{\log(p)}{p^{n/2}}}\cos(ns \log(p))$$ 

very visibly spikes at the zeros of zeta!

For fun, I just tried 

$$-\sum_{p^n} {\frac{\log(p)}{p^n}}\cos(ns \log(p))$$ 

and it spikes even more clearly at the zeros of zeta. 

\section{August 23, 2009}

Things to not forget:
\begin{enumerate}
\item Thanks to Sage and everybody who has worked on it.
\item NSF grant(s)
\item Students who read drafts and gave feedback. 
\end{enumerate}

\section{August 25, 2009}

Revisiting understanding why 


$$-\sum_{p^n} {\frac{\log(p)}{p^n}}\cos(ns \log(p))$$ 

"spikes" at the zeros of zeta.

The obvious first thing to do is think through where this came from in the first place.







\end{document}
