\documentclass{beamer}

\usepackage{fix-cm}
\usepackage{soul}
\usepackage{float}
\usepackage{tikz}

\usepackage{color}
\definecolor{dblackcolor}{rgb}{0.0,0.0,0.0}
\definecolor{dbluecolor}{rgb}{.01,.02,0.7}
\definecolor{dredcolor}{rgb}{0.8,0,0}
\definecolor{dgraycolor}{rgb}{0.30,0.3,0.30}
\usepackage{listings}
\lstdefinelanguage{Sage}[]{Python}
{morekeywords={True,False,sage,singular},
sensitive=true}
\lstset{frame=none,
          showtabs=False,
          showspaces=False,
          showstringspaces=False,
          commentstyle={\ttfamily\color{dredcolor}},
          keywordstyle={\ttfamily\color{dbluecolor}\bfseries},
          stringstyle ={\ttfamily\color{dgraycolor}\bfseries},
          language = Sage,
          basicstyle={\scriptsize \ttfamily},
          aboveskip=.3em,
          belowskip=.1em
          }

\usepackage{fancybox}
\usepackage{graphicx}
\usepackage{amsmath}
\usepackage{amsfonts}
\usepackage{amssymb}
\usepackage{amsthm}
\usepackage{url}


\DeclareMathOperator{\Gap}{Gap}
\DeclareMathOperator{\Li}{Li}
\DeclareGraphicsRule{.tif}{png}{.png}{`convert #1 `dirname #1`/`basename #1 .tif`.png}

\newcommand{\mycaption}[1]{\begin{quote}{\bf Figure: } \large #1\end{quote}}

\newcommand{\ill}[3]{%
   \begin{figure}[H]%
   \vspace{-2ex}
   \centering%
   \includegraphics[width=#2\textwidth]{illustrations/#1}%
   \caption{#3}%
   \vspace{-2ex}
    \end{figure}}

\newcommand{\illtwo}[4]{%
   \begin{figure}[H]\centering%
   \includegraphics[width=#3\textwidth]{illustrations/#1}$\qquad$\includegraphics[width=#3\textwidth]{illustrations/#2}%
   \caption{#4}%
    \end{figure}}

\newcommand{\illthree}[5]{%
   \begin{figure}[H]%
\centering%
   \includegraphics[width=#4\textwidth]{illustrations/#1}$\qquad$\includegraphics[width=#4\textwidth]{illustrations/#2}$\qquad$\includegraphics[width=#4\textwidth]{illustrations/#3}%
   \caption{#5}%
    \end{figure}}




\def\GL{\mathrm{GL}}
\def\PGL{\mathrm{PGL}}
\def\PSL{\mathrm{PSL}}
\def\GSP{\mathrm{GSP}}
\def\Z{\mathrm{Z}}
\def\Q{\mathrm{Q}}
\def\Gal{\mathrm{Gal}}
\def\Hom{\mathrm{Hom}}
\def\Ind{\mathrm{Ind}}
\def\End{\mathrm{End}}
\def\Aut{\mathrm{Aut}}
\def\loc{\mathrm{loc}}
\def\glob{\mathrm{glob}}
\def\Kbar{{\bar K}}
\def\D{{\mathcal D}}
\def\L{{\mathcal L}}
\def\R{{\mathcal R}}
\def\G{{\mathcal G}}
\def\W{{\mathcal W}}
\def\H{{\mathcal H}}
\def\OH{{\mathcal OH}}



\newcommand{\RH}{Riemann Hypothesis\index{Riemann Hypothesis}}


\title{PRIMES}
\author{Barry Mazur}
\date{\today}

\begin{document}

\begin{frame}
\titlepage

{\it (A discussion of `Primes: What is Riemann's Hypothesis?,' the book I'm currently writing with William Stein)}
\end{frame}

\begin{frame}
\frametitle{William:}
 (Here I'll put the video)
\end{frame}
\begin{frame}

\frametitle{The impact of the Riemann Hypothesis}
\ill{sarnak}{0.20}{Peter Sarnak}

\begin{quote}
``The Riemann hypothesis is the central problem and it implies many,
many things. One thing that makes it rather unusual in mathematics
today is that there must be over five hundred papers---somebody should
go and count---which start `Assume the Riemann hypothesis,' and
the conclusion is fantastic. And those [conclusions] would then become
theorems ... With this one solution you would have proven five hundred
theorems or more at once.'' 
\end{quote}

\end{frame}
\frametitle{An expository challenge}
     The approach you take depends upon your intended audiences. 
\end{frame}
\begin{frame}
\begin{frame}

\illtwo{factor_tree_300_a}{factor_tree_300_b}{.47}


\end{frame}

\subsection{Subsection no.1.1  }
\begin{frame}
Without title somethink is missing.
\end{frame}




\end{document}
%sagemathcloud={"zoom_width":105}